\title{Project Proposal\\Project Title: Implementing \textit{Frogger} Using Functional Programming}
\author{
  G54IPP\\
  Jack Ellis\\
  psyje5@nottingham.ac.uk\\
  4262333
}
\date{}
\documentclass[10pt]{article}
\usepackage{graphicx}
\graphicspath{ {Images/} }
\usepackage{mathtools}
\usepackage{pgfgantt}

\begin{document}
\maketitle

\subsection*{Background and Motivation}
\subsubsection*{Background}
Functional Programming (FP) has been increasingly adoped in industry, now to the extent that there is an entire conference dedicated to discussing its' use\cite{cufp}.
Currently, FP is predominantly used in the financial and research sectors, with only a couple of small game studios dedicated to using it.
I am interested in extending the use of Functional Programming to games and plan to dedicate my project to re-implementing a popular game, \textit{Frogger}, using Functional Programming.

\textit{Frogger} is a video game originally released in 1981 that sees a single player attempt to direct a a number of frogs (3,5, or 7 depending on difficulty) to their home via a busy road and a river full of hazards.
The player can only move forward or side-to-side, and has a number of lives defined by the number of frogs they have remaining.
Being originally an arcade game there is no defined end-point, however the game gets progressively harder as the user progresses until inevitably they fail.
I chose \textit{Frogger} because I believe it is well-suited to an FP implementation; the constantly looping traffic and dangers will, in my view, be able to be replicated using recursive function calls rather than large \verb|for| loops.

\subsubsection*{Motivation}
In my experience with the G51PGP (Programming Paradigms) and G52AFP (Advanced Functional Programming) modules, Haskell is a concise and powerful language, more so than imperative or Object-Oriented (OO) equivalents.
From this experience I believe that if more programs can be written using the Functional Programming paradigm, they will be more extensible, more maintainable, and more readily understandable (primarily due to the strong typing and the fact that they can be proved correct mathematically as opposed to being only tested, e.g. by way of some automated testing suite).


\subsection*{Aims and Objectives}
The aim of this project is to recreate \textit{Frogger}, as faithfully as possible (initially), using purely functional programming.

The key objectives of this project are:
\begin{enumerate}
  \item To establish appropriate mapping methodologies, using \textit{Frogger} as an example.
  \item To establish a set of criteria for comparisons of two \textit{Frogger} implementations.\\
        This will be done in two key areas:
    \begin{itemize}
      \item Software Architecture\\
        Does the Functional Programming implementation essentially imitate the exact nature of the Object-Oriented one?
      \item User Experience\\
        Does the Functional Programming implementation correctly mimic the function of the Object-Oriented one?
    \end{itemize}
  \item To design and conduct a user study in which a user will compare two implementations on an experience level.
  \item To demonstrate benefits of Functional Programming by either extending or refining the implementation.
\end{enumerate}

\subsection*{External Aspect}
Dr. Henrik Nilsson is a lecturer at the University of Nottingham whose main area of work is in Functional Reactive Programming\cite{henrikssite}, an area of Functional Programming dedicated to programming with respect to time\cite{frp}.
Per the opening section, I believe this project to have a significant external value, largely by demonstrating the utility of Functional Programming beyond analytical or research tools.
To this end, upon the project's conclusion I will release the source code on GitHub for public use.

\subsection*{Work Plan}
\subsubsection*{Methodologies}
The bulk of the project will be spent researching and determining the correct combination of specifications and requirements for the project.
This will involve an extensive study of \textit{Frogger} to determine the most important components of the game and gameplay experience, and will be conducted throughout the Autumn term.
The study will take two main forms: (1) investigating source code and (2) reading around the game in terms of strategy guides and articles detailing the actual gameplay experience.
In addition, I will be investigating proper testing methodologies, both on a software and user level, following Software Engineering principles (e.g. the V-Model) and applying them to Functional Programming.
Spring Term will be spent implementing the concepts and ideas researched.

\subsubsection*{Proposed Schedule of Work}
I will split the work into a series of packages:
\begin{enumerate}
  \item Analysis of the source code and mapping out the structure of the Frogger OO implementation.
  \item Mapping the OO implementation onto FP structure.
  \item Iterative implementation and testing of specific FP components.
  \item User study: recruitment, testing, feedback.
  \item Refinement of the implementation based on User study feedback.
  \item Documentation and project management\\
        Here a number of documents will be created, including but not limited to:
        \begin{itemize}
          \item Specification and Requirements
          \item Implementation Plan
          \item User study Plan
          \item User Study Report
          \item Software Documentation
        \end{itemize}
\end{enumerate}

\begin{center}
\includegraphics[width=\textwidth]{GanttPreChrist.png}\\
\includegraphics[width=\textwidth]{GanttPostChrist.png}\\
\end{center}


\begin{thebibliography}{0}
  \bibitem{cufp}
    Commercial Users of Functional Programming homepage
    http://cufp.org\\
    Accessed 2018-18-11
  \bibitem{henrikssite}
    Henrik Nilsson's personal website\\
    http://www.cs.nott.ac.uk/~psznhn/\\
    Accessed 2018-10-11
  \bibitem{frp}
    Functional Reactive Animation\\
    http://conal.net/papers/icfp97/\\
    Conal Elliott and Paul Hudak, 1997\\
    Accessed 2018-10-11
\end{thebibliography}


\end{document}
