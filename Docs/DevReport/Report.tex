\title{Development Report}
\author{
  \textbf{Author}\\
  Jack Ellis \\
  psyje5@nottingham.ac.uk\\
  4262333\\\\
  \textbf{Supervisor}\\
  Natasa Milic-Frayling\\
  psznm@nottingham.ac.uk
}
\date{}
\documentclass[12pt, a4paper]{report}
\usepackage{graphicx}
\usepackage{listings}
\usepackage{mathtools}
\lstset{
  basicstyle=\ttfamily\small,
  language=Haskell
}

\graphicspath{ {Images/} }
\begin{document}
\maketitle

\tableofcontents
\pagebreak

\section{Drawing a window}

The first stage of development for the game is to draw a window.
Per the specifications set out in my interim report, this window is to be 640 pixels wide by 480 pixels tall.
I first created a file called \verb|Main.hs|; this can be thought of as analagous to the \verb|main| function often found in imperative languages as the "point of entry" for a given program.
Here I first declared it as a module called \verb|Main|, and then declared the \verb|main| function itself.
Again similar to an imperative language, this is the function that will be called once the program is compiled and run.
It is a function of type \verb|IO()|, meaning that ultimately all it does is perform some \verb|IO| action, some action with a side effect; it takes no arguments and returns no value that can be passed into another function.
These functions often consist of multiple commands chained in a manner that could look imperative to some however, as we shall see, this is not necessarily the case.
To begin with we must initialise GLUT; this allows it to begin working and set up a session with the window system in use.
The function \verb|getArgsAndInitialize| also returns a tuple, with a string representing argument 0 (the program name) as the first element, and a list of strings representing the remaining arguments as the second.
To again liken it to an imperative language, this is the \verb|(int argc, char* argv)| seen so often in C and C-style \verb|main| functions.

\par

Now that the window session is initialised we must provide it with some parameters with which to draw a window.
In my motivation for doing this I said that Haskell does not provide users with the ability to declare global variables.
The GLUT library however does, via a module called \verb|StateVar|.
The library has a number of such \verb|StateVar| variables, many of which will be explored as this program becomes more and more complex, but for the moment the one that is of use to us is the \verb|initialWindowSize| variable.
This is of type \verb|StateVar Size|, where \verb|Size| is a type defined in the GLUT library as little more than a wrapper for two numbers - a more readily understood tuple.
We can use the \verb|$=| operator as a "setter" for \verb|StateVar| variables, and with it we can set the window's initial size to 640x480 pixels.

\par

With the required settings now correct we can call \verb|createWindow|, passing the string "Frogger" as the window's title, and the program should compile and run.
It does not, however, because we are missing something.
If we try to compile the code we get the following error:

\begin{lstlisting}[language=sh, xleftmargin=-0.1\textwidth, xrightmargin=-0.1\textwidth]
GLUT Warning: The following is a new check for GLUT 3.0; update your code.
GLUT Fatal Error: redisplay needed for window 1, but no display callback.
\end{lstlisting}

We need to set a \verb|DisplayCallback|, essentially a function describing how to draw things within the created window at each refresh.
In order to keep things clean this function will be declared in its own module, \verb|Display|.
Currently we're only interested in drawing the window in the first place, so we can just tell it to first clear the \verb|ColorBuffer| - setting the whole content of the window to black - and then \verb|flush|, which empties all command buffers, executing the commands as quickly as possible so as to ensure the image is as expected by the user.

\par

\begin{figure}[ht]
  \centering
  \caption{The blank window drawn}
  \includegraphics[width=0.5\textwidth]{DrawingAWindow.png}
  \label{fig:drawingawindow}
\end{figure}


With that done, we now have a window being drawn!
It is however just a blank, black-filled window\ref{fig:drawingawindow}.
In the next section we will add the game's titular \textit{Frogger}.
The full code is available in the appendix.

\appendix

\section{Drawing a window}
\subsection{Main.hs}
\begin{lstlisting}
module Main where

import Graphics.UI.GLUT
import Display

main :: IO()
main = do (_progName, _args) <- getArgsAndInitialize
          initialWindowSize $= Size 640 480
          createWindow "Frogger"
          displayCallback $= display
          mainLoop
\end{lstlisting}

\subsection{Display.hs}
\begin{lstlisting}
module Display (display) where

import Graphics.UI.GLUT

display :: DisplayCallback
display = do clear [ColorBuffer]
             flush
\end{lstlisting}
%\begin{thebibliography}{0}
%\end{thebibliography}

\end{document}
