\title{Specifications and Requirements for the proposed \textit{Frogger} implementation}
\author{
  Jack Ellis \\
  psyje5@nottingham.ac.uk\\
  4262333
}
\date{}
\documentclass[12pt]{article}
\usepackage{graphicx}
\graphicspath{ {Images/} }
\usepackage{mathtools}

\begin{document}
\maketitle

\section{Specifications}
\begin{itemize}
  \item The game shall take place in a 640x480 pixel (VGA resolution) window.
  \item The game shall be written in Haskell.
  \item The game shall make use of the OpenGL window system.
\end{itemize}

\section{Requirements}
\begin{itemize}
  \item The user shall control a single character on the screen (hereafter the "Frogger") using their computer keyboard.
  \item The objective of the game shall be to move from the start point to one of some number of goals via a series of moving obstacles spread across 10 "lanes".
  \item These lanes will consist of 5 road lanes and 5 river lanes, with a safe place between them.
    \begin{itemize}
      \item The road lane shall contain vehicles: cars and lorries.
      \item The river lane shall contain three types of obstacle:
        \begin{itemize}
          \item Logs: The basic object of the river, these shall provide safe passage however shall carry the Frogger in whichever direction they are moving.
          \item Turtles: Shall behave identically to the logs, however shall submerge themselves on a regular cycle.
          \item Crocodiles: Shall behave similarly to the logs, however if the Frogger steps onto their head they shall die.
        \end{itemize}
    \end{itemize}
  \item If the Frogger is hit by a vehicle, moves off the edge of the screen, or touches the water the player shall die, lose a life, and have to begin the level anew.
  \item The game shall have no defined end-point, only becoming progressively more difficult as it goes on by way of obstacles moving more quickly.
\end{itemize}

\end{document}
