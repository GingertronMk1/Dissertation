\title{Functional Frogger Information Sheet}
\author{
  Jack Ellis \\
  psyje5@nottingham.ac.uk\\
  4262333\\\\
}
\date{}
\documentclass[12pt, a4paper]{report}
\usepackage{graphicx}
\graphicspath{{Images/}}

\usepackage{listings}
\lstset{
  basicstyle=\ttfamily
}

\begin{document}
\maketitle

This information sheet describes information about this study, and the game of Frogger as a whole, before describing how data from the study will be used.

\section{Frogger}
\textit{Frogger} was first released by Konami in 1981 as an arcade machine.
Since then it has become the premiere "road-crossing simulator", with over a dozen official sequels.
The basic objective of the original game is to navigate a colony of frogs (the eponymous Froggers) to their homes, via a busy road and a river filled with hazards.
There are so many of these hazards, in fact, that in a review of the 1982 Atari port, \textit{Softline Magazine} stated that it had "earned the ominous distinction of being the arcade game with the most ways to die". Indeed, a player can die from a number of mistimed jumps, world events, and even by reaching the goal at the wrong time.

\par

The game is highly popular, having sold in excess of 20 million copies, and there have been countless imitations created right from its inception; from 1982's \textit{Ribbit} for the Apple II, to the iOS and Android app \textit{Crossy Road} essentially aping the "road-crossing simulator" aspect of the original arcade game.

\par

The controls are simple, consisting solely of button presses allowing the Frogger to jump one unit up, down, left, or right.

\section{This Study}
This study has one key aim: to determine how similar an extant version of Frogger is to a port, developed in the Haskell programming language.
The study will likely take around 30 minutes, broken down as follows:
\begin{itemize}
  \item 10 minutes: playing one version of the game
  \item 5 minutes: make notes about that version
  \item 10 minutes: playing the other version of the game
  \item 5 minutes: fill out a short survey rating various elements of the two games out of 10, 10 being identical and 0 being completely different
\end{itemize}

\section{Prize Draw}
If you choose to, you can provide your email address as part of this study.
If you do so you will be entered into a prize draw to win a free drink at the Portland Coffee Company.
Any personal information you do provide will only be used to contact you in the event of your winning this giveaway, and will be destroyed immediately after the prize draw.

\section{How I will use information from this study}
Information from this study will be used to determine how successful the porting of Frogger to Haskell was, on an experiential level.
This will then hopefully be used to discuss what aspects of a game should be considered the most vital in porting that game between platforms.

\section{Further Information}
If you have any further questions, or would like to discuss aspects of this study further, please contact Jack Ellis (psyje5@nottingham.ac.uk) for more information.

\end{document}

