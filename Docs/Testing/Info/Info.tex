\title{Functional Frogger Information Sheet}
\author{
  Jack Ellis \\
  psyje5@nottingham.ac.uk\\
  4262333\\\\
}
\date{}
\documentclass[12pt, a4paper]{report}
\usepackage{graphicx}
\graphicspath{{Images/}}

\usepackage{listings}
\lstset{
  basicstyle=\ttfamily
}

\begin{document}
\maketitle

This information sheet describes the study of the game of Frogger that we are conducting, and the potential reward that you may get for playing the game.

\section{Frogger}
\textit{Frogger} was first released by Konami in 1981 as an arcade machine.
Since then it has become the premiere "road-crossing simulator", with over a dozen official sequels.
In the game the player navigates a colony of frogs (the eponymous Froggers) to their homes, via a busy road and a river filled with hazards.
There are so many of these hazards, in fact, that in a review of the 1982 Atari port, \textit{Softline Magazine} stated that it had "earned the ominous distinction of being the arcade game with the most ways to die".
Indeed, a player can die from a number of mistimed jumps, world events, and even by reaching the goal at the wrong time.

\par

The game is highly popular, having sold in excess of 20 million copies, and there have been countless imitations created right from its inception; from 1982's \textit{Ribbit} for the Apple II, to the iOS and Android app \textit{Crossy Road} essentially aping the "road-crossing simulator" aspect of the original arcade game.

\par

The controls are simple, consisting solely of button presses allowing the Frogger to jump one unit up, down, left, or right.

\section{This Study}
In this study you will be asked to play a game of Frogger twice and provide feedback on whether there are any differences between the two game sessions.
The study will likely take around 30 minutes:
\begin{itemize}
  \item 10 minutes: First play session.
  \item 5 minutes: If you need to, make notes about your experience on the provided sheet to help you complete the survey (these notes will not be collected).
  \item 10 minutes: Second play session.
  \item 5 minutes: If you need to, make notes about your experience on the provided sheet to help you complete the survey (these notes will not be collected).
  \item 5 minutes: Fill out a short survey rating various aspects of the two play sessions for their similarity.
        Assign the score on the scale 0-10 on the scale 0-10, 10 being identical and 0 being completely different.
\end{itemize}

\par

You will be provided with a copy of the survey at the beginning of the survey that you can fill out after the second session.

\section{Prize Draw}
If you choose to, you can provide your email address as part of this study and enter into a prize draw to win a free drink at the Portland Coffee Company.
Any personal information you do provide will only be used to contact you in the event of your winning this giveaway.
The data will be destroyed immediately after the prize draw.

\section{How information from this study will be used}
None of the information from this study will be disclosed publically.
All information will be used according to the consent form.

\section{Further Information}
If you have any further questions, or would like to discuss aspects of this study further, please contact Jack Ellis (psyje5@nottingham.ac.uk) for more information.

\end{document}
